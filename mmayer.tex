
%%%%%%%%%%%%%%%%%
% This is an sample CV template created using altacv.cls
% (v1.7.4, 30 Jul 2025) written by LianTze Lim (liantze@gmail.com), based on the
% CV created by BusinessInsider at http://www.businessinsider.my/a-sample-resume-for-marissa-mayer-2016-7/?r=US&IR=T
%
%% It may be distributed and/or modified under the
%% conditions of the LaTeX Project Public License, either version 1.3
%% of this license or (at your option) any later version.
%% The latest version of this license is in
%%    http://www.latex-project.org/lppl.txt
%% and version 1.3 or later is part of all distributions of LaTeX
%% version 2003/12/01 or later.
%%%%%%%%%%%%%%%%

%% Use the "normalphoto" option if you want a normal photo instead of cropped to a circle
% \documentclass[10pt,a4paper,withhypeper,normalphoto]{altacv}

\documentclass[10pt,a4paper,withhyper]{altacv}
%% AltaCV uses the fontawesome5 and simpleicons packages.
%% See http://texdoc.net/pkg/fontawesome5 and http://texdoc.net/pkg/simpleicons for full list of symbols.
\usepackage{hyperref}
% Change the page layout if you need to
\geometry{left=1.25cm,right=1.25cm,top=1.5cm,bottom=1.5cm,columnsep=1.2cm}

% The paracol package lets you typeset columns of text in parallel
\usepackage{paracol}


% Change the font if you want to, depending on whether
% you're using pdflatex or xelatex/lualatex
% WHEN COMPILING WITH XELATEX PLEASE USE
% xelatex -shell-escape -output-driver="xdvipdfmx -z 0" mmayer.tex
\iftutex
  % If using xelatex or lualatex:
  \setmainfont{Lato}
\else
  % If using pdflatex:
  \usepackage[default]{lato}
\fi

% Change the colours if you want to
\definecolor{VividPurple}{HTML}{000000}
\definecolor{SlateGrey}{HTML}{2E2E2E}
\definecolor{LightGrey}{HTML}{666666}
% \colorlet{name}{black}
% \colorlet{tagline}{PastelRed}
\colorlet{heading}{VividPurple}
\colorlet{headingrule}{VividPurple}
% \colorlet{subheading}{PastelRed}
\colorlet{accent}{VividPurple}
\colorlet{emphasis}{SlateGrey}
\colorlet{body}{LightGrey}

% Change some fonts, if necessary
% \renewcommand{\namefont}{\Huge\rmfamily\bfseries}
% \renewcommand{\personalinfofont}{\footnotesize}
% \renewcommand{\cvsectionfont}{\LARGE\rmfamily\bfseries}
% \renewcommand{\cvsubsectionfont}{\large\bfseries}

% Change the bullets for itemize and rating marker
% for \cvskill if you want to
\renewcommand{\cvItemMarker}{{\small\textbullet}}
\renewcommand{\cvRatingMarker}{\faCircle}
% ...and the markers for the date/location for \cvevent
% \renewcommand{\cvDateMarker}{\faCalendar*[regular]}
% \renewcommand{\cvLocationMarker}{\faMapMarker*}


% If your CV/résumé is in a language other than English,
% then you probably want to change these so that when you
% copy-paste from the PDF or run pdftotext, the location
% and date marker icons for \cvevent will paste as correct
% translations. For example Spanish:
% \renewcommand{\locationname}{Ubicación}
% \renewcommand{\datename}{Fecha}


%% Use (and optionally edit if necessary) this .tex if you
%% want to use an author-year reference style like APA(6)
%% for your publication list
% \input{pubs-authoryear.cfg}

%% Use (and optionally edit if necessary) this .tex if you
%% want an originally numerical reference style like IEEE
%% for your publication list
\input{pubs-num.cfg}

%% sample.bib contains your publications
\addbibresource{sample.bib}

\begin{document}
\name{Yahia Abdeldjallil BENAMROUCHE}
% Cropped to square from https://en.wikipedia.org/wiki/Marissa_Mayer#/media/File:Marissa_Mayer_May_2014_(cropped).jpg, CC-BY 2.0
%% You can add multiple photos on the left or right
\photoR{2.5cm}{mmayer-wikipedia-cc-by-2_0}
% \photoL{2cm}{Yacht_High,Suitcase_High}
\personalinfo{%
  % Not all of these are required!
  \email{yahia.benamrouche@gmail.com}
  \phone{+213(0)542449732}
  \mailaddress{UV1,Alimendjeli , Constantine , Algerie
}
  \linkedin{yah04dev}
  % \twitter{@marissamayer}
  \github{@yah04dev}
\rule{\textwidth}{0.4pt}
\begin{center}
\textbf \larh{
Étudiant en L3 informatique à l’Université de Constantine 2, passionné par l’informatique depuis l’enfance et motivé par l’aide aux autres, je prévois après la fin de mon L3 de poursuivre un master en bio-informatique pour réunir ces deux passions.}
\end{center}
\rule{\textwidth}{0.4pt}
%   \github{github.com/mmayer} % I'm just making this up though.
%   \orcid{0000-0000-0000-0000} % Obviously making this up too.
  %% You can add your own arbitrary detail with
  %% \printinfo{symbol}{detail}[optional hyperlink prefix]
  %% \printinfo{\faPaw}{Hey ho!}
  %% Or you can declare your own field with
  %% \NewInfoFiled{fieldname}{symbol}[optional hyperlink prefix] and use it:
  % \NewInfoField{gitlab}{\faGitlab}[https://gitlab.com/]
  % \gitlab{your_id}
	%%
  %% For services and platforms like Mastodon where there isn't a
  %% straightforward relation between the user ID/nickname and the hyperlink,
  %% you can use \printinfo directly e.g.
  % \printinfo{\faMastodon}{@username@instace}[https://instance.url/@username]
  %% But if you absolutely want to create new dedicated info fields for
  %% such platforms, then use \NewInfoField* with a star:
  % \NewInfoField*{mastodon}{\faMastodon}
  %% then you can use \mastodon, with TWO arguments where the 2nd argument is
  %% the full hyperlink.
  % \mastodon{@username@instance}{https://instance.url/@username}
}

\makecvheader

%% Depending on your tastes, you may want to make fonts of itemize environments slightly smaller
\AtBeginEnvironment{itemize}{\small}

%% Set the left/right column width ratio to 6:4.
\columnratio{0.6}

% Start a 2-column paracol. Both the left and right columns will automatically
% break across pages if things get too long.
\begin{paracol}{2}


\cvsection{Formation}

\cvevent{Licence en Sciences de l’informatique}{Université de Constantine 2}{Septembre 2023 -- En cours}

\small{ $\bullet$ Animé d’un fort intérêt pour l’informatique, j’ai validé avec succès les deux premières années de licence, acquérant une solide base Je souhaite désormais approfondir mes connaissances en poursuivant ma formation en L3.} 



\divider

\cvevent{Bac math technique option “Genie Civil” (candidat libre)}{Direction de l’Éducation de Constantine}{Juillet 2023}
\small {$\bullet$ Après avoir constaté que mes premiers résultats au baccalauréat ne me permettaient pas de m’orienter vers l’informatique, j’ai choisi de le repasser en candidat libre pour suivre la voie qui me correspond .}

\divider

\cvevent{Bac math technique option “Genie Civil”}{Lycée Benabdelkader Med Elarbi}{Septembre 2019 -- Juillet 2022}

\small{ $\bullet$ J’ai étudié pendant trois ans dans un lycée technique, dans un cursus basé sur les mathématiques, la physique et le génie civil. J’ai ensuite obtenu mon baccalauréat avec une moyenne de 12.90} 


\cvsection{Les projets}
\cvevent{Hospito}{\href{https://github.com/yah04dev/hospito}{https://github.com/yah04dev/hospito}}{mai 2023 -- Juillet 2023}

\small{ $\bullet$ Application web développée avec Flask et SQLite permettant la gestion des patients, des médecins ...}


\cvsection{Travail bénévole}
\cvevent{Volontaire
}{Croissant-Rouge Algérien}{mai 2024 -- Juillet 2025 }

\small{ $\bullet$ J’ai rejoint le Croissant-Rouge comme bénévole par passion pour l'aide aux autres. J'ai participé à des initiatives telles que la distribution de nourriture et l'organisation de collectes de sang, contribuant ainsi à ma communauté.
}

\newpage
\cvsection{Programme d'échange}
\cvevent{Programme numérique de leadership jeunesse et de service communautaire}{World Learning (The Experiment Digital)
}{juin 2022 -- Aout 2022 }

\small{ $\bullet$ Un programme d’échange en ligne de deux mois m’a permis de développer de solides compétences en leadership, citoyenneté et gestion de projet, enrichissant ainsi mon parcours personnel.
}


\cvsection{Certifications}
\cvevent{Introduction à l’impact de l’IA générative sur la transformation des soins de santé}{Linkedin Learning}{Aout 2025}

\small{ $\bullet$ Curieux de voir comment la bio-informatique transformera radicalement les soins de santé, j’ai suivi ce cours comme introduction afin de comprendre comment l’informatique contribuera à ce changement
}

\cvevent{Introduction à l’intelligence artificielle }{Linkedin Learning}{Juillet 2025}

\small{ $\bullet$ Curieux de l’IA, j’ai suivi ce cours pour acquérir les bases et mieux comprendre le vocabulaire des vidéos, podcasts et ateliers que je regarde habituellement.}

\cvevent{Atelier  “Flutter”}{Institut français Constantine
}{Janvier 2025}

\small{ $\bullet$ J’ai participé à un workshop mobile à l’Institut français, par curiosité et pour découvrir comment développer une application native.
}

\cvevent{Secouriste certifié “AFPS”}{Croissant-Rouge Algérien}{Janvier 2024}

\small{ $\bullet$ Une belle expérience m'a appris à faire correctement les premiers secours (RCP, manœuvre de Heimlich..)
}






\switchcolumn

\color{black}
\cvsection{Languages}
\begin{itemize}
\item{Arabe : langue maternelle} %% supports X.5 values.

\item{Francais: B2 (DELF)}

\item{English : B2 (EF SET)}

\item{Espagnol : A1}

\end{itemize}
\cvsection{Compétences}
\begin{itemize}
\item{Python} %% supports X.5 values.

\item{Flask}

\item{HTML/CSS/JS}

\item{Arduino / Esp32 (IOT)}

\item{SQL}
\end{itemize}
\cvsection{Centres d’intérêt}
\begin{itemize}

\item{Humanité}

\item{La psychologie}

\item{La baignade}

\item{L’exploration des paysages }
\end{itemize}
\end{paracol}

\end{document}
